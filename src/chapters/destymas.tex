\section{Medžiagos darbo tema dėstymo skyriai}

% Medžiagos darbo tema dėstymo skyriuose išsamiai pateikiamos nagrinėjamos temos detalės: pradiniai duomenys, jų analizės ir apdorojimo metodai, sprendimų įgyvendinimas, gautų rezultatų apibendrinimas.

% Medžiaga turi būti dėstoma aiškiai, pateikiant argumentus. Tekste dėstomas trečiuoju asmeniu, t.y. rašoma ne „aš manau“, bet „autorius mano“, „autoriaus nuomone“. Reikėtų vengti informacijos nesuteikiančių frazių, pvz., „...kaip jau buvo minėta...“, „...kaip visiems žinoma...“ ir pan., vengti grožinės literatūros ar publicistinio stiliaus, gausių metaforų ar panašių meninės išraiškos priemonių.

% ALT + 0150