\sectionnonumnocontent{Santrauka}

Šiame darbe nagrinėjamos IOTA platformos panaudojimo galimybės konkrečiuose tiekimo grandinės procesuose. 
Darbe analizuojama tiekimo grandinių dalykinė sritis, jos sąvoka, struktūra ir šiandieninės problemos.
Aprašyta išskirstyto transakcijų žurnalo technologija, jai keliami reikalavimai tiekimo grandinių kontekste. Išanalizuotos technologijos atmainos, paremtos blokų grandinių ir orientuoto grafo be ciklų principais, pristatytos jų savybės, privalumai ir trūkumai. 
Sukonstruotas pavyzdinis tiekimo grandinės modelis ir diskretūs IOTA platformos taikymo atvejų pavyzdžiai konkrečiuose etapuose, skirti pagrįsti platformos panaudojamumą tiekimo grandinėse. Pateiktos taikymo alternatyvos. 
Darbe pasiūlytos potencialios sistemos svarbiausios užduotys ir veiklos, reikalingos praktinio taikymo įgyvendinimui.

\raktiniaizodziai{Išskirstytas transakcijų žurnalas, blokų grandinė, IOTA, tiekimo grandinė}


% Glaustai aprašomas darbo turinys: pristatoma nagrinėta problema ir padarytos išvados. Santraukos apimtis ne didesnė nei 0,5 puslapio. Santraukų gale nurodomi darbo raktiniai žodžiai. 
% Nurodomi iki 5 svarbiausių temos raktinių žodžių (terminų).
% Vienas terminas gali susidėti iš kelių žodžių.

% Pirmame skyriuje buvo atlikta tiekimo grandinės dalykinės srities analizė: apibrėžta sąvoka, struktūra ir problemoms. Antrajame skyriuje analizuot išskirstyto transakcijų žurnalo technologija, jai keliami reikalavimai iš dalykinės srities pusės, analizuotos technologijos atmainos ir jų savybės, o atmainos palygintos. Remiantis blokų grandinės ir IOTA atmainų savybių palyginimu, pasirinkta viena iš technologijos atmainų, autoriaus vertinimu tinkamesnė taikymui tiekimo grandinėse. Trečiajame skyriuje sumodeliuotas ir pateiktas tiekimo grandinės pavyzdinis atvejis su detaliu paaiškinimu, kiekvienas modelio etapas papildytas IOTA technologinių savybių pritaikymu, konkretūs pavyzdžiai detaliai paaiškinti. Taip pat pristatyti alternatyvūs scenarijai ir taikymo pavyzdžiai, išanalizuotos potencialiai sistemai reikalingos atlikti užduotys, detaliai atvaizduotos užduočių ir veiksmų diagramose.