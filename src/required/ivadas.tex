% --------------------------------------------------------------- %
%                               ĮVADAS                         
% --------------------------------------------------------------- %

\sectionnonum{Įvadas}

Modernią visuomenę sunku įsivaizduoti be nuolatinio prekių, paslaugų ir informacijos judėjimo. Prekės transportuojamos iš vienos vietos į kitą, internetu perduodami milžiniški srautai informacijos, atliekamos transakcijos tarp skirtingų verslo šalių. Tačiau natūraliai pamatyti šiuos procesus vis dėlto nėra toks lengvas uždavinys. Visi šie procesai yra paslėpti po sudėtinga verslo logika, modeliais ir technologijomis. 

Apskritai šie sudėtingi procesai kuria milžinišką pridėtinę vertę, o kartu gerina ir bendrą pasaulinį ekonomikos lygį. Pavyzdžiui, logistika sudaro 7,5\% JAV BVP\footnote{Informacija paimta iš: https://www.atkearney.com/transportation-travel/article?/a/2017-state-of-logistics-report-article. Tikrinta 2019-05-14.}, o duomenų srautai 2014 metais prie pasaulinio BVP lygio prisidėjo \$2.8 trln. JAV dolerių \cite{manyika2016digital}. Todėl, siekiant verslo progreso ir konkurencinio pranašumo, šiuos procesus stengiamasi kuo labiau optimizuoti bei automatizuoti. Atrodo, kad tai davė savo vaisių. Per pastaruosius dešimtmečius žmonija tapo gerokai pažangesnė: visuomenės kompiuterinis raštingumas išaugo \cite{van2015internet}, o kartu skaitmenizavosi ir verslas – pradėtos naudoti sudėtingos IT sistemos ir sprendimai.

Pasaulyje esant per 7,5 mlrd. žmonių ir šiam skaičiui vis dar augant\footnote{Informacija paimta iš: http://www.worldometers.info/world-population. Tikrinta 2019-05-14.}\footnote{\label{note1}Informacija paimta iš: https://www.google.com/publicdata/directory. Tikrinta 2019-05-14.} bei gerėjant ekonominėms sąlygoms, kartu nenumaldomai auga ir vartojimas\footnote{2012-2017 metais visuotinis BVP padidėjo nuo \$70 iki \$80 trln. JAV dolerių. Duomenys paimti iš https://www.google.com/publicdata/directory. Tikrinta 2019-05-14.}, o klientai kelia vis aukštesnius reikalavimus \cite{nilsson2006logistics}. Taigi, norėdamos išlikti konkurencingomis, įmonės varžosi tarpusavyje, naudoja įvairias laiko optimizavimo strategijas, užtikrinančias paslaugų greitį \cite{zacharia2004logistics}, investuoja į vis pažangesnes informacines technologijas. Visa tai tam, kad kad produktas iš gamintojo į pirkėjo rankas patektų kuo kokybiškiau ir optimaliau.

Paprastai šis produkto gyvavimo ciklas, nuo pradinio tiekėjo iki galutinio kliento rankų, yra vadinamas tiekimo grandine. Tačiau tai nenusako, kokios apimties ir sudėtingumo logistika vyrauja tiekimo grandinėse. Vien 2017 metais buvo pasiekta apie 750 milijonų TEU standarto jūrinių konteinerių krova\footnotemark[\ref{note1}], o tais pačiais metais Klaipėdos uosto metinė krova viršijo 43 mln. tonų\footnote{Informacija paimta iš: https://sumin.lrv.lt/lt/naujienos/klaipedos-uostas-lyderis-regione. Tikrinta 2019-05-14.}. Ir tai tik jūrų krovinių dalis. Remiantis Transparency Market Research duomenimis, iki 2023 metų pajamos visoje logistikos rinkoje turėtų išaugti iki \$15,5 trln. JAV dolerių, o krova iki 92 mlrd. tonų\footnote{Informacija paimta iš: https://www.prnewswire.com/news-releases/global-logistics-market-to-reach-us155-trillion-by-2023-research-report-published-by-transparency-market-research-597595561.html. Tikrinta 2019-05-14.}.

Šie duomenys nepaprastai svarbūs, nes leidžia suprasti, kokią naudą visai žmonijai ir galimybes rinkai gali atverti IT sprendimų sėkmingas taikymas tiekimo grandinėse. Tiesa, įmonės, užsiimančios logistika ir tiekimo grandinių valdymu, tą iš dalies jau atlieka. Tai programinė įranga ir technologijos, tokios kaip CRM \cite{bharati2015current}, ERP \cite{neubert2018collaboration} ir daugelis kitų. Tačiau visi jie turi savų trūkumų ir nėra tinkami kiekvienai tiekimo grandinei \cite{garg2018supply}.

Yra teigiama, kad šiandieninėms problemoms spręsti tiekimo grandinėse į pagalbą gali ateiti daiktų internetas \cite{dweekat2017supply} ir iki tol dar plačiai netaikyta išskirstyto transakcijų žurnalo technologija \cite{abeyratne2016blockchain}. Daiktų internetas padėtų sekti produktų būseną gyvavimo ciklo metu – tai šiuo metu naudojami RFID prietaisai, galintys perduoti informaciją realiu laiku \cite{majeed2017internet}. Tuo tarpu išskirstyto transakcijų žurnalo technologijos plačiau žinomos dėl vienos iš atšakų – blokų grandinės. Tačiau vis didesnes perspektyvas rodo IOTA platforma \cite{popov2016tangle}, paremta kita atšaka – orientuotais grafais be ciklų, ir teigianti, jog yra pritaikyta būtent daiktų internetui ir tiekimo grandinėms. 

%------------------ OBJEKTAS ------------------
Tyrimo objektas – išskirstyto transakcijų žurnalo technologijos vienos iš atmainų panaudojimas tiekimo grandinėje. Šiuo metu tai yra jauna technologija, neturinti daug įgyvendintų pripažintų pavyzdžių tiekimo grandinių procesuose. Taip pat nėra aišku, kuri technologijos atmaina yra parankesnė ir kokiais konkrečiais būdais įgalinti technologiją.

%------------------ TIKSLAS -------------------
\textbf{Darbo tikslas} – pateikti išskirstyto transakcijų žurnalo technologijos sprendimus tiekimo grandinės procesuose. 

%----------------- UŽDAVINIAI -------------–---
\textbf{Darbo uždaviniai}:
\begin{enumerate}
    \item Apžvelgti tiekimo grandinės dalykinę sritį: jos svarbiausias sąvokas, struktūrą ir naudojamas technologijas;
    \item Išanalizuoti išskirstyto transakcijų žurnalo technologiją ir jos atmainas: blokų grandinę ir orientuotą grafą be ciklų, juos palyginti;
    \item Sukonstruoti pavyzdinį tiekimo grandinės modelį ir, taikant vieną iš išskirstyto transakcijų žurnalo technologijos atmainų, jį atnaujinti;
    \item Pateikti potencialios sistemos esmines užduotis ir veiklas.
\end{enumerate}