\sectionnonum{Įvadas}

XXI a. būtų sunku įsivaizduoti pasaulį be nuolatinio prekių, paslaugų ir informacijos judėjimo. Prekės pastoviai transportuojamos iš vienos vietos į kitą, internetu perduodami milžiniški srautai informacijos, atliekamos transakcijos tarp skirtingų verslo šalių. Tačiau natūraliai pamatyti šiuos procesus vis dėlto nėra toks lengvas uždavinys. Visi šie procesai yra paslėpti po sudėtinga verslo logika, modeliais ir technologijomis. 

Apskritai šie sudėtingi procesai kuria milžinišką pridėtinę vertę, o kartu gerina ir bendrą pasaulinį ekonomikos lygį. Pavyzdžiui, logistika sudaro 7,5\% JAV BVP\footnote{https://www.atkearney.com/transportation-travel/article?/a/2017-state-of-logistics-report-article}, o duomenų srautai 2014 metais prie pasaulinio BVP lygio prisidėjo \$2.8 trln. JAV dolerių \cite{manyika2016digital}. Todėl, siekiant verslo progreso ir konkurencinio pranašumo, šie procesai buvo ir yra optimizuojami bei automatizuojami [REF]. Atrodo, kad tai davė savo vaisių. Per pastaruosius dešimtmečius žmonija tapo smarkiai pažangesnė [REF]. Prekes transportuoja iki 60 tonų vežantys krovininiai sunkvežimiai \cite{lumsden2004truck} ir per 21 tūkst. TEU standarto konteinerių sugebantys plukdyti krovininiai laivai \cite{halff2019likely}, visuomenės kompiuterinis raštingumas išaugo \cite{van2015internet}, o kartu skaitmenizavosi ir verslas - pradėtos naudoti sudėtingos IT sistemos ir sprendimai [REF].

Pasaulyje esant per 7,5 mlrd. žmonių ir šiam skaičiui vis dar augant\footnote{http://www.worldometers.info/world-population. Tikrinta 2019-02-01}\footnote{https://www.google.com/publicdata/directory. Tikrinta 2019-02-01} bei gerėjant ekonominėms sąlygoms, kartu nenumaldomai auga ir vartojimas (2012-2017 metais visuotinis BVP padidėjo nuo \$70 iki \$80 trln. JAV dolerių\footnote{https://www.google.com/publicdata/directory. Tikrinta 2019-02-01}), o klientai kelia vis aukštesnius reikalavimus \cite{nilsson2006logistics}. Taigi, norėdamos išlikti konkurencingomis, įmonės varžosi tarpusavyje, naudoja įvairias laiko optimizavimo strategijas, užtikrinančias paslaugų greitį \cite{zacharia2004logistics}, investuoja į vis pažangesnes informacines technologijas [REF]. Visa tai tam, kad kad produktas iš gamintojo į pirkėjo rankas patektų kuo kokybiškiau ir optimaliau.

Paprastai šis produkto gyvavimo ciklas, nuo tiekėjo iki pat kliento rankų, yra vadinamas tiekimo grandine [REF]. Tiekimo grandinę formaliai galime apibūdinti ir kaip dalyvaujančių organizacijų tinklą, kuris skirtingais procesais ir veiklomis kuria vertę produktų ir paslaugų pavidalu vartotojui \cite{christopher2016logistics}. Tačiau šie apibrėžimai nenusako, kokio masto ir sudėtingumo logistika ir kiti tarpiniai procesai vyrauja tiekimo grandinėse. Vien 2017 metais buvo pasiekta apie 750 milijonų TEU standarto jūrinių konteinerių krova\footnote{https://data.worldbank.org/indicator/IS.SHP.GOOD.TU. Tikrinta 2019-02-01}, o tais pačiais metais Klaipėdos uosto metinė krova viršijo 43 mln. tonų\footnote{https://sumin.lrv.lt/lt/naujienos/klaipedos-uostas-lyderis-regione. Tikrinta 2019-02-03}. Ir tai tik jūrų krovinių dalis. Remiantis Transparency Market Research duomenimis, iki 2023 metų pajamos visoje logistikos rinkoje turėtų išaugti iki \$15,5 trln. JAV dolerių, o krova iki 92 mlrd. tonų\footnote{https://www.prnewswire.com/news-releases/global-logistics-market-to-reach-us155-trillion-by-2023-research-report-published-by-transparency-market-research-597595561.html. Tikrinta 2019-02-01}.

Šie duomenys nepaprastai svarbūs, nes leidžia suprasti temos aktualumą, t.y. kokią naudą visai žmonijai ir galimybes rinkai gali atverti informacinių technologijų pritaikymas tiekimo grandinėse. Tiesa, įmonės, užsiimančios logistika ir tiekimo grandinių valdymu, jau taiko nemažai informacinių technologijų. Tai programinė įranga ir technologiniai sprendimai, tokie kaip CRM \cite{bharati2015current}, ERP \cite{neubert2018collaboration} ir daugelis kitų. Tačiau visi jie turi savų trūkumų ir nėra tinkami kiekvienai tiekimo grandinei \cite{garg2018supply}.

Bet yra manančių, kad šiandieninėms problemoms spręsti tiekimo grandinėse į pagalbą gali ateiti daiktų internetas \cite{dweekat2017supply} ir iki tol dar plačiai netaikytos paskirstytos didžiaknygės technologijos \cite{abeyratne2016blockchain}. Daiktų internetas padėtų sekti produktų būseną gyvavimo ciklo metu - tai šiuo metu naudojami RFID prietaisai, galintys perduoti informaciją realiu laiku \cite{majeed2017internet}. Tuo tarpu paskirstytų didžiaknygių technologijos plačiau žinomos dėl vienos iš atšakų - blokų grandinės, tačiau didesnes perspektyvas rodo IOTA technologija \cite{popov2016tangle}, paremta kita paskirstytų didžiaknygių technologijos atšaka - DAG ir teigianti, jog yra pritaikyta būtent šiai probleminei sričiai [REF]. 

IOTA ne tik pasižymi naudingomis paskirstytų didžiaknygių technologijos savybėmis \cite{bramas2018stability}, bet ir leidžia efektyviai išnaudoti daiktų internetą [REF]. Tai yra nauja technologija, tačiau jei pavyktų ją sėkmingai pritaikyti praktikoje, ji galėtų tapti inovacija, atnaujinanti procesus, vykstančius kiekvieną dieną ir nuo kurių priklauso mūsų gyvenimo kokybė.

%------------------ TIKSLAS -------------------
 
Šiuo darbu siekiama įvertinti ir sumodeliuoti galimus pokyčius tradiciniuose tiekimo grandinės modeliuose pradėjus naudoti paskirstytų didžiaknygių technologijas.
Šiam tikslui pasiekti keliami tokie uždaviniai:

%----------------- UŽDAVINIAI -----------------

\begin{itemize}
    \item Apžvelgti tiekimo grandinės dalykinę sritį, jos svarbiausias sąvokas, modelius ir naudojamas technologijas;
    \item Ištirti paskirstytos didžiaknygės technologijų idėją ir atmainas: blokų grandinę ir IOTA, jas palyginti;
    \item Sukonstruoti tiekimo grandinės modelį pritaikius IOTA technologiją ir palyginti jį su tradiciniu modeliu.
\end{itemize}

% Įvade apibrėžiamas tiriamasis objektas akcentuojant neapibrėžtumą, kuris bus išspręstas darbe, aprašomas temos aktualumas, nurodomas darbo tikslas ir uždaviniai, kuriais bus įgyvendinamas tikslas, aptariamos teorinės darbo prielaidos bei metodika, apibūdinami su tema susiję literatūros ar kitokie šaltiniai, temos analizės tvarka, darbo atlikimo aplinkybės, pateikiama žinių apie naudojamus instrumentus (programas ir kt., jei darbe yra eksperimentinė dalis). Darbo įvadas neturi būti dėstymo santrauka. Įvado apimtis 2 -- 4 puslapiai.