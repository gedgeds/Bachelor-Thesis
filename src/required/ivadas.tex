\sectionnonum{Įvadas}
Įvade apibrėžiamas tiriamasis objektas akcentuojant neapibrėžtumą, kuris
bus išspręstas darbe, aprašomas temos aktualumas, nurodomas darbo tikslas
ir uždaviniai, kuriais bus įgyvendinamas tikslas, aptariamos teorinės darbo prielaidos
bei metodika, apibūdinami su tema susiję literatūros ar kitokie šaltiniai,
temos analizės tvarka, darbo atlikimo aplinkybės, pateikiama žinių apie
naudojamus instrumentus (programas ir kt., jei darbe yra eksperimentinė dalis).
Darbo įvadas neturi būti dėstymo santrauka. Įvado apimtis 2 -- 4 puslapiai.