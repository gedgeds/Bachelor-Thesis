% --------------------------------------------------------------- %
%                              SĄVOKOS                           
% --------------------------------------------------------------- %

\sectionnonum{Sąvokų apibrėžimai}

\noindent \textbf{51\% ataka} - vienas iš išskirstyto transakcijų žurnalo atakos būdų, kuomet kenkėjas bando turėti daugiau nei 50\% viso tinklo pajėgumų ir šitaip jį užvaldyti.

\noindent \textbf{Atlikto darbo konsensuso algoritmas} - konsensuso algoritmas, paremtas dalyvavimu tinkle atliekant skaičiavimus.

\noindent \textbf{Bitcoin} - pirmoji blokų grandinės principu paremta platforma, pristatyta 2008 metais.

\noindent \textbf{Blokas} - blokų grandinių architektūrinis vienetas ir duomenų struktūra, laikanti savyje kitus duomenis.

\noindent \textbf{Blokų grandinė} - Viena iš išskirstytų transakcijų technologijos atmainų, kuri savyje laiko duomenis, tarpusavyje sujungtus į nuoseklių blokų seką.

\noindent \textbf{CRM} - ryšių su klientais valdymo sistema, padedanti organizuoti ir valdyti visos įmonės darbą, nukreiptą į esamų ir potencialių klientų poreikių patenkinimą. 

\noindent \textbf{Decentralizuota programėlė} - programinis kodas, vykdomas išskirstytose sistemose.

\noindent \textbf{Dvigubo išleidimo problema} - potenciali spraga skaitmeninių pinigų sistemoje, kuomet tas pats piniginis vienetas gali būti išleistas ir priklausyti daugiau nei vienam asmeniui tuo pačiu metu.

\noindent \textbf{Ekonominis klasteris} - grupė IOTA tinklo narių, esančių tame pačiame regione.

\noindent \textbf{ERP} - programinė įranga, skirta kompiuterizuoti įmonės valdymą apjungiant duomenis ir procesus joje.

\noindent \textbf{Ethereum} - blokų grandinės principu paremta platforma, pirmą kartą pristatyta 2014 metais.

\noindent \textbf{GAS limitas} - Ethereum platformos vienetas, skirtas nusakyti, maksimalų GAS, kurį transakcijos kūrėjas yra linkęs išleisti už transakcijos patvirtinimą tinkle.

\noindent \textbf{GDPR reglamentas} - Europos Parlamento ir Europos Tarybos priimtas visoje ES tiesiogiai taikomas teisės aktas, įgyvendinantis asmens duomenų apsaugos reformą.

\noindent \textbf{IOTA} - orientuotų grafų be ciklų principu paremta platforma, pirmą kartą pristatyta 2015 metais.

\noindent \textbf{IOTA raizginys} - IOTA tinklo apibūdinimas.

\noindent \textbf{Išmanusis kontraktas} - programinis kodas, automatiškai vykdantis komandas pagal prieš tai aprašytas taisykles.

\noindent \textbf{Išskirstyto transakcijų žurnalo technologija} - duomenų bazė, kuria konsensuso būdu dalinasi ir operuoja skirtingi naudotojai tinkle.

\noindent \textbf{Kasėjas} - asmuo, atliekantis skaičiavimus blokų grandinėse su tikslu sukurti bloką ir už tai gauti kriptovaliutos atlygį.

\noindent \textbf{Konsensuso algoritmas} - protokolas, kuris pasirūpina, kad visi tinklo nariai sinchronizuotųsi tarpusavyje ir prieitų bendrą sutarimą.

\noindent \textbf{Kriptopiniginė} – skaitmeninė piniginė, skirta kriptovaliutų siuntimui, gavimui ir laikymui.

\noindent \textbf{Kriptovaliuta} - skaitmeninis turtas, naudojamas išskirstytų transakcijų žurnalų transakcijose.

\noindent \textbf{Maišos reikšmė} - tam tikro ilgio skaitinė reikšmė, skirta identifikuoti unikalius duomenis.

\noindent \textbf{MAM kanalas} - specialus darinys IOTA tinkle, leidžiantis sukurti duomenų srautą, kurį gali prenumeruoti kiti asmenys tinkle.

\noindent \textbf{Markov Chain Monte Carlo algoritmas} - algoritmas, skirtas pasirinkti ir patvirtinti IOTA tinkle esančias viršūnes.

\noindent \textbf{Maskuotieji nustatytos tapatybės pranešimai} - biblioteka, užšifruojanti, iššifruojanti ir nustatanti tapatybę duomenų, kuriuos yra norima publikuoti į IOTA raizginį.

\noindent \textbf{Orakulas} - IOTA tinklo tarpininkas su išoriniu pasauliu.

\noindent \textbf{Orientuotas grafas be ciklų} - grafas, kurio visos briaunos turi kryptį ir savyje neturintis ciklų.

\noindent \textbf{Panaudos atvejo diagrama} - UML diagrama, apibūdinanti, ką projektuojama sistema gali atlikti, kartu aprašydama ir išorinius sistemos veikėjus. 

\noindent \textbf{Qubic protokolas} - protokolas, veikiantis kaip atskiras IOTA sluoksnis, įgalinantis išmaniuosius kontraktus, orakulus ir išskirstytus skaičiavimus.

\noindent \textbf{Raizginio viršūnė} - IOTA raizginio transakcija, kurios nėra patvirtinusi jokia kita transakcija.

\noindent \textbf{RFID prietaisai} - priemonės, skirtos radijo dažnio bangų pagalba siųsti žinutes.

\noindent \textbf{Stiprinanti transakcija} - speciali transakcija IOTA tinkle, kuriama naudotojo savo paties transakcijų patvirtinimo tinkle tikimybei padidinti.

\noindent \textbf{TEU standartas} - standartinis vienetas, paremtas ISO 20 pėdų ilgio konteineriu ir naudojamas kaip statistinė eismo srauto ar mato priemonė.

\noindent \textbf{Tiekimo grandinė} - organizacijų, procesų, finansų, informacijos ir kitų esybių visuma, dalyvaujanti produkto gyvavimo cikle nuo pradinio tiekėjo iki galutinio kliento.

\noindent \textbf{Transakcijos kaupiamasis svoris} - IOTA transakcijos atributas, nusakantis šios transakcijos ir visų kitų transakcijų, tiesiogiai arba netiesiogiai patvirtinančių šią transakciją, svorių suma.

\noindent \textbf{Transakcijos per sekundę} - dydis, skirtas nusakyti, kiek transakcijų arba įrašų atsiduria transakcijų žurnale per sekundę.

\noindent \textbf{Transakcijos svoris} - IOTA transakcijos atributas, nusakantis, kiek tinklo naudotojas įdėjo pastangų sukurdamas transakciją.

\noindent \textbf{Transakcijos taškai} - IOTA transakcijos atributas, nusakantis visų tiesiogiai arba netiesiogiai šios transakcijos patvirtintų kitų transakcijų ir kartu šios transakcijos svorių sumą.

\noindent \textbf{Turimos įtakos konsensuso algoritmas} - konsensuso algoritmas, paremtas turimos įtakos tinkle, pavyzdžiu kriptovaliuta, disponavimu.

\noindent \textbf{Veiklų UML diagrama} - UML diagrama, aprašanti konkretaus scenarijaus vykdomus veiksmus.

\noindent \textbf{Viršūnių parinkimo algoritmas} - algoritmas, skirtas IOTA tinklo dalyviams atsirinkti transakcijas, kurios bus patvirtintos.

\noindent \textbf{Winternitz vienkartinė parašo panaudojimo schema} - kvantiniams kompiuteriams atspari schema, skirta generuoti viešus raktus IOTA tinkle.