\sectionnonum{Sąvokų apibrėžimai}

\noindent \textbf{51\% ataka} -

\noindent \textbf{Atlikto darbo konsensuso algoritmas} -

\noindent \textbf{Bitcoin} -

\noindent \textbf{Blokas} -

\noindent \textbf{Blokų grandinė} -

\noindent \textbf{CRM} -

\noindent \textbf{Decentralizuota programėlė} -

\noindent \textbf{Dvigubo išleidimo problema} -

\noindent \textbf{Ekonominis klasterizavimas} -

\noindent \textbf{ERP} -

\noindent \textbf{Ethereum} -

\noindent \textbf{GAS limitas} -

\noindent \textbf{GDPR reglamentas} -

\noindent \textbf{IOTA} -

\noindent \textbf{IOTA raizginys} -

\noindent \textbf{Išmanusis kontraktas} -

\noindent \textbf{Išskirstyto transakcijų žurnalo technologija} -

\noindent \textbf{Kasėjas} -

\noindent \textbf{Konsensuso algoritmas} -

\noindent \textbf{Kriptopiniginė} -

\noindent \textbf{Kriptovaliuta} -

\noindent \textbf{Kvorumu paremti skaičiavimai} -

\noindent \textbf{Maišos reikšmė} -

\noindent \textbf{MAM kanalas} -

\noindent \textbf{Markov Chain Monte Carlo algoritmas} -

\noindent \textbf{Maskuotieji nustatytos tapatybės pranešimai} -

\noindent \textbf{Mikrotransakcija} -

\noindent \textbf{Orakulas} -

\noindent \textbf{Orientuotas grafas be ciklų} -

\noindent \textbf{Panaudos atvejo UML diagrama} -

\noindent \textbf{Privatus ir viešas raktai} -

\noindent \textbf{Qubic protokolas} -

\noindent \textbf{Raizginio viršūnė} -

\noindent \textbf{RFID prietaisai} -

\noindent \textbf{Stiprinanti transakcija} -

\noindent \textbf{TEU standartas} -

\noindent \textbf{Tiekimo grandinė} -

\noindent \textbf{Transakcijos kaupiamasis svoris} -

\noindent \textbf{Transakcijos per sekundę} -

\noindent \textbf{Transakcijos svoris} -

\noindent \textbf{Transakcijos taškai} -

\noindent \textbf{Transakcijų žurnalas} -

\noindent \textbf{Turimos įtakos konsensuso algoritmas} -

\noindent \textbf{UML diagrama} -

\noindent \textbf{Veiklų UML diagrama} -

\noindent \textbf{Viršūnių parinkimo algoritmas} -

\noindent \textbf{Winternitz vienkartinė parašo panaudojimo schema} -